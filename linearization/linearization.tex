\documentclass[12pt]{article}
\usepackage{fullpage,enumitem,amsmath,amssymb,graphicx,grffile,float,listings}



\begin{document}
    %Title Section
    \begin{flushleft}
	\LARGE Dynamical Systems: \\
	Linearization, Equilibrium Points \& Stability
    \end{flushleft} 
    \rule{\linewidth}{0.4pt}
    %Title Section

    %Problem and Solution

    \section*{Equilibrium Points}

	Consider the following nonlinear system:
        $$\dot q_1 = aq_1 - bq_1q_2$$
		$$\dot q_2 = bq_1q_2-cq_2$$
	where $q_1, q_2 \geq 0$ and $a, b, c \geq 0$ are positive constants.
	This can be represented as:
		$$\frac{dx}{dt} = f(x) \text{;}$$	
		$$x = \begin{bmatrix} q_1 \\ q_2 \end{bmatrix}$$
		$$\frac{dx}{dt} = \begin{bmatrix} \dot q_1 \\ \dot q_2 \end{bmatrix}$$
		$$f(x) = \begin{bmatrix} aq_1 - bq_1q_2 \\ bq_1q_2-cq_2 \end{bmatrix}$$
	Equilibrium points occur when our system does not change with time. This occurs when:
		$$\dot q_1 = 0$$
		$$\dot q_2 = 0$$
	Therefore, equilibrium points occur when:
		$$q_1 (a-bq_2) = 0$$
		$$q_2 (bq_1-c) = 0$$
	We can solve and show that there are two equilbrium points:
	\begin{itemize}
		\item Equilibrium Point 1: 
			$$q_1=0 \text{ , } q_2=0 $$

		\item Equilibrium Point 2: 
			$$q_1= \frac{c}{b} \text{ , } q_2= \frac{a}{b}$$
	\end{itemize}

	\newpage

	\section*{Jacobian Linearization Around an Equilibrium Point}
	
	The Jacobian linearization of our nonlinear system is:
		$$\frac{dx}{dt} = Ax$$
	where $A$ is the Jacobian of our matrix $f(x)$.
	Let $f_1 = \dot q_1$, $f_2 = \dot q_2$, and $x = \begin{bmatrix} q_1 \\ q_2 \end{bmatrix}$.

		$$A = \frac{\partial f}{\partial x} =
		\begin{bmatrix}
		\frac{\partial f_1}{\partial q_1} &	\frac{\partial f_1}{\partial q_2} \\
		\\
		\frac{\partial f_2}{\partial q_1} &	\frac{\partial f_2}{\partial q_2}
		\end{bmatrix}$$
		
	\begin{itemize}
		\item Evaluate at Equilibrium Point \#1:
	
			$$A = \begin{bmatrix} a-bq_2 & -bq_1 \\ bq_2 & bq_1-c \end{bmatrix} $$

			At Equilibrium Point 1, $q_1 = 0$ and $q_2 = 0$.

				$$A = \begin{bmatrix} a & 0 \\ 0 & -c \end{bmatrix} $$
				$$det(A-\lambda I) = 0$$

			Therefore, $\lambda_1 = a$ and $\lambda_2 = -c$.
			Since $\lambda_1$ is positive, the system is unstable around equilibrium point \#1.

		\item Evaluate at Equilibrium Point \#2:
			$$A = \begin{bmatrix} a-bq_2 & -bq_1 \\ bq_2 & bq_1-c \end{bmatrix} $$

			At Equilibrium Point 2, $q_1 = \frac{c}{b}$ and $q_2 = \frac{a}{b}$.
				$$A = \begin{bmatrix} 0 & -c \\ a & 0 \end{bmatrix} $$
				$$det(A-\lambda I) = 0$$

			Therefore, $\lambda_1 = +i\sqrt{ac}$ and $\lambda_2 = -i\sqrt{ac}$. Since $\lambda_1$ and $\lambda_2$ are complex conjugates, the system is marginally stable around this equilibrium point.
	\end{itemize}

	\newpage

	\section*{Simulation of System}

	Choosing arbitrary constants $a, b, c$, we can simulate the response of this system for varying initial conditions. We choose a small time step, $\Delta t$, and assume that $f$ is constant during the interval. Under these assumptions, $q$ can be iteratively updated by
	$$q_{t + \Delta t} = q_t + \Delta t \dot{q_t} = q_t + \Delta t f(q)$$
	We choose a step size of $\Delta T = 0.1$, set constants $a = b = c = 1$, and plot the trajectory of $q_1$ vs $q_2$ over 20s for the following initial conditions:
	\begin{itemize}
		\item $q_1 = 1$, $q_2 = 3$
		\item $q_1 = 1.1$, $q_2 = 1$
		\item $q_1 = 1$, $q_2 = 1.5$
	\end{itemize}

\end{document}
